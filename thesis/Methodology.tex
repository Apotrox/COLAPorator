\chapter{Methodology}
\label{sec:Methodology}

(Overall, i kept it similar to the Methodology from the proposal and removed parts that just weren't relevant as the proposal was intended for an entirely different purpose than this project ended up having. I changed some parts that were relevant to be more accurate. Though i feel like this is just too little?)\\\\



In this section, I will outline the iterative design-implementation-evaluation approach employed, to develop an interactive poster system using Raspberry Pi as the central microcontroller. The methodology is rooted in user-centered design principles, emphasizing collaboration with stakeholders, iterative refinement, and requirements-driven evaluation to ensure the system meets its intended purposes.

The primary goal is to create a system that presents the COLAPS group's research in a concise and engaging way. The system should balance technical functionality with usability, visual appeal, and ease of integration into various contexts, such as education, exhibitions, or public spaces.

My approach begins with identifying key stakeholders and eliciting requirements through interviews and contextual inquiries. These requirements guide the creation of initial design mockups, which visualize the interaction concepts and serve as a foundation for stakeholder feedback. Based on these designs, prototypes are developed incrementally, with each iteration refining the system's hardware, software, and user experience.

Each prototype undergoes systematic evaluation, combining qualitative and quantitative methods to assess its technical performance, usability, and alignment with stakeholder needs. The insights gained from these evaluations inform subsequent iterations, fostering continuous improvement and innovation.

By following this methodology, I aim to ensure the final system not only meets technical and aesthetic expectations but also achieves its intended impact, creating a versatile, interactive platform that bridges the gap between traditional and digital media.

\section{Identifying Stakeholders}
In the development of this project, it is crucial to identify and engage the right stakeholders to ensure the system meets their diverse needs and expectations. Depending on who they are, the course of the entire project might change, thus this should be done as the very first step.
\begin{itemize}
	\item End Users (General Audience): The primary users of the interactive paper poster will be individuals who engage with the poster. These users could interact with the poster through buttons (capacitive or tactile) or sensors, triggering visual or mechanical responses (e.g. lighting, moving parts, screen display). Their experience with this system will be central to its success. As such, their feedback on usability, engagement, and overall experience is critical for shaping the design and functionality.
	\item  Research group: On the other side of the poster presentations are the presenters themselves, the research group \emph{colaps}. They will be the ones showcasing the poster and presenting their research using it. As such, the system needs to be reliable and easy to use.
\end{itemize}

\section{Eliciting Requirements}
The process of eliciting requirements will be carried out to ensure that the interactive paper poster meets the needs of its primary stakeholders. This phase will involve gathering information about the expectations, preferences, technical constraints, and contextual constraints, that would shape the design and functionality of the poster.
This could be done by interviewing the team of the research group to get a grasp on the factors mentioned, as well as collecting and surveying the necessary research material that is desired to be presented.
When starting the prototyping process, designing mockups, and implementing various features, a strong feedback loop with the research group should be established to guarantee that no unnecessary features will be implemented or important features will be left out.


\section{Prototyping and Evaluating}
In the process of prototyping, mockups of the poster will be made, to test certain features and to obtain feedback on their implementation. Each prototype will then be evaluated according to the abovementioned fundamental requirements, the functionality of the feature(s) implemented, and the impressions and feedback of the research group.\\\\


\section{Materials}
At the most fundamental level, a Raspberry Pi SBC or similar will be required for this project. For this, I propose using a Raspberry Pi 4 2GB. The reason for this is that, compared to the Pi 3 B+ for example, the Pi 4 offers two Micro HDMI and USB 3 ports, offering great expansion capabilities in terms of screens and IO. Its more powerful processor and ample RAM provide a more reliable basis for computing without the risk of running into any potential bottlenecks.
Additionally, materials need to be researched that fulfill the requirements set by the stakeholders. \textbf{(Maybe elaborate more on the material research here? There wasn't \textit{a lot} but i think it's worth mentioning on how i went about this? Or is this more implementation?)}\\\\


(I'd like to keep this in to then refer to in the results. Could be interesting to see how risks were actively handled once/after they occurred)\\

\section{Risk Assessment Plan}

The development of the interactive paper poster system may involve several risks that need to be identified and managed to ensure the project stays on track. The table below outlines the primary risks that can be identified at this point in time, as well as mitigation strategies.

\begin{table}[h!]
	\centering
	\caption{Risk Assessment Plan}
	\begin{tabular}{|p{4cm}|p{8cm}|}
		\hline
		\textbf{Risk} & \textbf{Mitigation Strategy} \\
		\hline
		\textbf{Stakeholder Miscommunication} & Establish clear communication channels (e.g., regular meetings, documented feedback) and maintain a feedback loop with stakeholders throughout the project. \\
		\hline
		\textbf{Hardware Failures or Delays} & Maintain spare components for critical hardware (e.g., Raspberry Pi, sensors, and peripherals) and plan for alternatives in case of supply chain issues. \\
		\hline
		\textbf{Technical Integration Challenges} & Adopt a modular design approach to minimize dependencies between components, allowing easier troubleshooting and replacement of individual parts. \\
		\hline
		\textbf{Time Overruns Due to Feature Creep} & Strictly prioritize features based on stakeholder requirements and time constraints; establish a clear Minimum Viable Product (MVP). \\
		\hline
		\textbf{Insufficient Usability Feedback} & Conduct frequent usability testing with diverse stakeholders at every iteration to ensure feedback is gathered systematically. \\
		\hline
		\textbf{Software Bugs or System Instability} & Implement version control (e.g., Git) and perform continuous integration and testing to catch and resolve bugs early. \\
		\hline
		\textbf{Power or Portability Issues} & Test power requirements early in the prototyping phase and optimize for efficient power consumption. Use battery packs or alternative power sources as backups. \\
		\hline
		\textbf{Unfamiliarity with Certain Technologies} & Allocate time for researching and experimenting with new technologies (e.g., advanced sensors or peripherals) during the early phases of the project. Seek guidance from experts if needed. \\
		\hline
	\end{tabular}
	\label{tab:risk-assessment}
\end{table}