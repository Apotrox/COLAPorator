\chapter{Related Works}
\label{sec:Related_Work}

\section{Background and Limitations of Traditional Posters}
(Citation on what are posters, maybe \cite{Soon}?: "Posters are displayed on boards or stands and viewed from a distance. Posters are displayed simultaneously over the course of the conference and attendees are free to browse and study at their own convenience" Or put this in the introduction?)\\\\

Traditional paper posters have long been used as a tool for conveying information, whether it be in educational, professional, or public settings \cite{Soon, IlicRowe3}. \enquote{The 'traditional' poster presentation aims to present information in a succinct manner}\cite{IlicRowe4}. However, despite their widespread use and popularity \cite{IlicRowe3}, simple paper posters suffer from significant limitations, particularly in their ability to actively engage the audience and their bandwidth for transferring knowledge \cite{IlicRowe1}. It was shown that while most event participants agreed that posters were a good medium for knowledge transfer \cite{Arslan, goodhand}, the knowledge retention of their viewers was often abysmal if they were read at all \cite{goodhand}. For example, at the Digestive Diseases Week (DDW) and British Society of Gastroenterology (BSG) meetings, only \textless1.5\% and \textless0.3\% read any of the posters respectively. \enquote{Delegates remembered very little when phoned about posters two weeks later} \cite{goodhand}. While it may be the case that \enquote{Publication rates of abstracts submitted to the BSG seem to be falling, implying that for more and more research, poster presentations are the only opportunity to share research findings} \cite{goodhand}. \enquote{Abstracts selected for presentation are usually collated as conference proceedings, but rarely found in the published literature.} \cite{Soon}. \\


These findings highlight a key challenge with traditional paper posters: their ability to captivate and engage an audience effectively \textbf{(citation needed)}. Given the limited time and attention that viewers devote to posters, their aesthetic appeal becomes a crucial factor in attracting viewers and encouraging interaction\cite{goodhand}. A visually striking design can often be the deciding factor in whether a poster is noticed, read, or remembered, emphasizing that aesthetics are not merely decorative but integral to the effectiveness of paper posters\cite{Erren, Arslan, Mabrouk}.
\enquote{Studies that reported on the effectiveness of the poster presentation as a standalone intervention were unanimous in their conclusions that the poster was not effective at facilitating knowledge transfer be it through an increase in knowledge, change in attitude or behavior. his conclusion was supported by an evaluation study, in which participants identified that posters needed to be accompanied by another source of information to be effective – otherwise the only drawing point to the poster is the imagery.} \cite{IlicRowe1} 
Factors like layout, format, readability, and even color schemes all influence how effectively key information can be conveyed to the reader \cite{IlicRowe1, Erren}.\\

\enquote{Given its passive nature; if not accompanied by an active intervention (e.g. oral presentation, physical interaction), which can help with aural and verbal learning exchange, the ‘traditional’ poster may only reach a limited proportion of its intended audience .} \cite{IlicRowe1}
This nature also makes them badly equipped to accommodate alternative learning styles.\cite{IlicRowe4} 

\enquote{By embedding knowledge in interactions that involve people, it is possible to achieve reciprocal dialogue, which is the most effective method of transferring tacit knowledge.} \cite{IlicRowe1}

\section{Evolution Toward Interactive Posters}
\subsection{Paper Posters}
These interactions don't need to be exclusively based on dialogue only; Posters that ask the viewer to interact with them in some way help to attract attention and might even instigate thoughtful debate among attendees \cite{Dale}. When asked for participation, participants spent longer at the poster, as it required them to think about the given questions/topic more thoroughly and deal with it in more detail \cite{Dale, Mabrouk}. This increased engagement also causes participants to return to the poster more frequently, even if it is just to compare their results with those of their peers \cite{Dale, Mabrouk}. This not only helps the participants to obtain more knowledge about a given topic but also the authors, as it identifies improved ways to communicate complex information \cite{Dale}. However, it is important to note that \enquote{Simplicity is vital so that attendees are not waiting for directions}\cite{Dale}, \enquote{one engaged viewer will attract others}\cite{Erren}.

\subsection{Digital Posters}
Digital posters, sometimes called E-Posters, are a form of digital media, used to present information, just like "traditional" posters \textbf{(citation needed)}. \enquote{Reported advantages of the digital format include lower production costs, ease of preparation and transport, dissemination to a larger audience, and large archival capabilities}\cite{Newsom.2021}. This also opens up the possibility of maneuvering the presented content in a 3D space as well as embedding multimedia elements, creating new ways of presenting information to an audience \cite{Venkatesan.2019}. Technologies like DIPP (digital interactive poster presentation)\cite{Simone.2001}, MediaPoster\cite{Rowe}, as well as ePoster, emerged, trying to make use of the flexibility digital media has to offer \textbf{(citation needed or reformulate)}. \\
\begin{itemize}
	\item DIPP (digital interactive poster presentation) \enquote{is a pdf version of a traditional poster that can be projected on a wall or screen at allotted times} \cite{Angelo}, which is mostly used to briefly summarize the research, giving the audience the chance to decide which presentations to attend during the actual poster sessions \cite{Angelo}.
	\item MediaPoster expands on this, adding the ability to embed external sources into an area of interest. Unlike normal hyperlinks, which would open a website or similar, a MediaPoster has a Media Display Area to the right, which would show the requested documents to the viewer, keeping them within the same environment with the poster always in full view. \cite{Rowe}
	\item ePoster by Conventus takes a slightly different approach, offering posters in digital format on dedicated hardware. Presentations are held on specialized screens, controlled by a device similar to a smartphone. The control device mirrors the large display but enables manipulation of the viewing area, as well as selecting different presentations, with its touchscreen. \textbf{(What to do here????)}
\end{itemize}	
\textbf{(removed the pictures here, i probably need the rights to use them?)}

This innovative approach to posters not only enhances the flexibility and interactivity of the presentation but also significantly improves the audience's ability to grasp and retain the material \textbf{(citation needed or reform)}. The integration of multimedia elements such as pictures and videos captures attention more effectively than traditional posters \cite{Sumantri}, serving as powerful tools for explaining complex concepts \textbf{(citation needed or reform)}. By embedding visual aids and interactive content, digital posters make the material more accessible and easier to understand, while also highlighting and clarifying essential parts of the presentation\cite{Sumantri}.

\section{Technological Foundations for Enhanced Interactivity}
\subsection{Electric Circuits and Other Responsive Media}

\textbf{(redo this section? its mostly claims...)}\\\\

Thanks to modern production and fabrication capabilities, building and integrating electric circuits has become more accessible. With the advent of tools like printed circuit boards (PCBs), conductive inks, and modular components, creating responsive and interactive systems is no longer limited to specialists. These advances have enabled the seamless integration of electric circuits into a variety of materials and media, opening the door to innovative applications in design, art, and engineering.

Among these innovations, responsive media such as Thermochromic ink, thin-film displays and a plethora of sensors have gained significant attention. Thermochromic inks, for example, can change color in response to temperature fluctuations, providing a visually dynamic way to convey information or interact with users. When paired with electric circuits, these materials create systems that are both functional and expressive, blending technology with creativity.
Projects like PaperPixels \cite{paperpixels} even go as far as to create Paper-based Displays with thermochromic ink by creating a circuit that powers piezoelectric elements according to an animation specified in their custom software, heating the ink at specific times. Due to the thermochromic ink's nature, they then change colors based on the temperature of the piezoelectric element. Depending on the "color scheme" chosen, images could appear "out of thin air" if the background color matches the unheated color of the ink.\\


Another notable project is IllumiPaper \cite{illumipaper}, which makes use of thin-film (TF) displays with electroluminescent or electrochromic properties, to light up predefined regions or change color based on user interactions. These displays are printed on the paper itself, creating a seamless integration of digital and physical media. Users can then interact with the media via a pen, using the Anoto technology to track the pen's movement pattern and capture input. This works by printing an "invisible" dot pattern on the paper which encodes specific coordinates, allowing the pen to determine its exact position on the paper.\\

\textbf{(Would love to put the picture here but i'd have to get the rights for that, right?)}


\textbf{(how to go about commercial solutions? hard to cite...)}\\\\

Commercial solutions to this also exist, notably, Interactive Paper \cite{interactivepaper} by a company with the same name, Interactive Paper. Their approach is electric circuits made out of conductive ink printed on the inside of the paper, sending NFC signals to a smartphone in contact with the respective NFC pad on the paper to then perform an action. They also offer a version that uses augmented reality to display content on a smartphone, though they have not published any information on the inner workings of this process.

\section{Interactive Media in Education}

\textbf{(citation needed for below claims)}
Active student engagement is a cornerstone of effective learning environments. When students are actively involved in their educational experience—through participation, interaction, and hands-on activities—they are more likely to retain knowledge, develop critical thinking skills, and cultivate a genuine enthusiasm for learning. Engagement transforms the classroom from a passive setting into a dynamic space where curiosity and creativity thrive.

Research consistently highlights the benefits of interactive and participatory learning methods \textbf{(citation needed)}. Students who feel connected to their learning materials and peers are more likely to exhibit improved academic performance\cite{Groccia}. This is particularly true when teaching methods leverage modern technologies, real-world applications, and collaborative activities that resonate with students’ interests and experiences\cite{Sahronih}.
\enquote{Even students who do not talk in class are often stimulated by questions or problem-solving exercises as they think about what they would answer in a particular situation}\cite{Steinert}, helping them become more actively involved with the material or content, teacher or even their peers \cite{Steinert}.
Interactive media will be more effective when already associated with high learning motivation \cite{Sahronih2}, resulting in students, who are already highly motivated to learn and study the material, receiving an even higher beneficial effect from this form of media \cite{Sahronih2}. Students with lower learning motivation, however, could suffer from this kind of learning, as they might require more intervention and more supervision from a teacher during the learning process \cite{Sahronih2}. Thus, it is important to choose an appropriate kind of media.