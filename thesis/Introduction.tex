\chapter{Introduction}
\label{sec:Introduction}

In academic and research settings, posters serve as a key medium for presenting work, fostering dialogue, and sparking collaboration. While effective in their simplicity, traditional paper posters often rely heavily on the physical presence of the researcher to guide discussions, answer questions, and provide context. This dependency limits their utility, particularly in contexts where accessibility, interactivity, or remote engagement are prioritized. The proposed thesis topic, "Implementation of an interactive poster using Raspberry Pi and traditional crafting materials," addresses this gap by exploring how modern technological tools can enhance the traditional research poster, transforming it into an interactive and self-explanatory artifact.

This project tries to bridge the fields of technology, design, and communication. By integrating microcontrollers such as Raspberry Pi with traditional crafting materials like paper and markers, the aim is to create a hybrid solution that maintains the familiarity of traditional posters while introducing elements of interactivity. %This interactive poster will serve as a proof-of-concept, demonstrating how users can engage with content independently through embedded functionality, animations, or even guided narratives.

The thesis will adopt an iterative design-implementation-evaluation approach, ensuring that the final prototype meets the needs of its stakeholders and demonstrates usability in real-world scenarios. Initial phases will involve identifying the primary users and their requirements through stakeholder analysis. Based on this foundation, mockups and prototypes will be developed and refined through user feedback and usability evaluations. Each step will build toward delivering a functional and engaging interactive poster that leverages technology to improve the accessibility and communication of research.

This project aligns with the broader goal of advancing educational and presentation technologies by introducing creative, user-centered approaches to traditional formats. By focusing on the process of integrating hardware, software, and crafting techniques, this thesis will not only contribute to the field of interactive learning technologies but also inspire further exploration into how technology can enhance communication in academic and professional contexts.